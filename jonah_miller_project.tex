\documentclass[]{article}
% Author: Jonah Miller (jonah.maxwell.miller@gmail.com)
% Time-stamp: <2013-12-11 01:21:27 (jonah)>

% packages
\usepackage{fullpage}
\usepackage{amsmath}
\usepackage{amssymb}
\usepackage{latexsym}
\usepackage{graphicx}
\usepackage{mathrsfs}
\usepackage{verbatim}
\usepackage{braket}
\usepackage{listings}
\usepackage{pdfpages}
\usepackage{listings}
\usepackage{color}
\usepackage{hyperref}

% for the picture environment
\setlength{\unitlength}{1cm}

%preamble
\title{Numerical Explorations of FLRW Cosmologies}

\author{Jonah Miller\\
  \textit{jonah.maxwell.miller@gmail.com}} 

\date{General Relativity for Cosmology\\Achim Kempf\\ Fall 2013}

% Macros
\newcommand{\R}{\mathbb{R}} % Real number line
\newcommand{\N}{\mathbb{N}} % Integers
\newcommand{\eval}{\biggr\rvert} %evaluated at
\newcommand{\myvec}[1]{\vec{#1}} % vectors for me
% total derivatives 
\newcommand{\diff}[2]{\frac{d #1}{d #2}} 
\newcommand{\dd}[1]{\frac{d}{d #1}}
% partial derivatives
\newcommand{\pd}[2]{\frac{\partial #1}{\partial #2}} 
\newcommand{\pdd}[1]{\frac{\partial}{\partial #1}} 
% t derivatives
\newcommand{\ddt}{\dd{t}}
\newcommand{\dt}[1]{\diff{#1}{t}}
% A 2-vector
\newcommand{\twovector}[2]{\left[\begin{matrix}#1\\#2\end{matrix}\right]}
% A 3 vector.
\newcommand{\threevector}[3]{\left[\begin{matrix} #1\\ #2\\ #3 \end{matrix}\right]}

\begin{document}

\maketitle

% \begin{abstract}
%   In this paper, we numerically investigate the past and future of the
%   night sky. We briefly discuss the origin of the so-called
%   Friedmann-Lemaitre-Robertson-Walker metric, the homogenous,
%   isotropic matter-filled universe. We numerically explore the
%   evolution of the metric given various equations of state of the
%   fluid filling spacetime.
% \end{abstract}

\section{Introduction}
\label{sec:intro}

To attain a picture of the universe as a whole, we must look at it on
the largest possible scales. Since the universe is almost certainly
much larger than the observable universe, or indeed the universe we
will ever observe
\cite{MapOfUniverse,Carroll,MisnerThorneWheeler,Wald}, some guesswork
is necessary \cite{Wald}. The most popular---assumption
used to study the universe as a whole is the \textit{Copernican
  principle} \cite{BonesOfCopernicus}, the notion that, on the larges
scales, the universe looks the same everywhere and in every direction
\cite{Carroll,Wald}.

In modern language, we assume that the universe is homogeneous and
isotropic. That is, we assume that the energy-momentum tensor that
feeds Einstein's equations is the same everywhere and in space and has
no preferred direction
\cite{Carroll,MisnerThorneWheeler,Wald}. Obviously, this isn't the
only approach one can take. Roughly speaking one can generate a
spacetime with an arbitrary amount of symmetry between no symmetry and
maximal symmetry. The homogeneous isotropic universe is almost as
symmetric as one can get. One can add time-like symmetry to attain de
Sitter, anti de Sitter, or Minkowski spacetimes, but that's as
symmetric as you can get
\cite{Carroll,MisnerThorneWheeler,Wald}. Relaxing the symmetry
conditions on the energy-momentum tensor results in, for example, the
mixmaster universe \cite{MisnerThorneWheeler}. However, observational
evidence, most strongly the cosmic microwave background, suggests that
the universe is homogeneous and isotropic to a very high degree
\cite{Planck}.

In this work, we derive the equations governing the evolution of a
homogeneous isotropic universe and numerically explore how such a
universe could evolve. In section \ref{sec:the:metric} we briefly
derive the so-called Friedmann equations, the equations governing such
spacetimes. In section \ref{sec:numerical:approach}, we briefly
discuss the numerical tools we use to evolve these equations. In
section \ref{sec:results}, we present our results. Finally, in section
\ref{sec:conclusion}, we offer some concluding remarks.

\section{The Friedmann-Lemaitre-Robertson-Walker Metric}
\label{sec:the:metric}

First, we derive the metric of a homogenous, isotropic spacetime. Then
we derive the analytic formulae for its evolution. Because we derived
the Friedmann equations using tetrads in class \cite{Kempf}, we
decided to work in coordinates for another perspective.

\subsection{Deriving the Metric}
\label{subsec:metric:derivation}

% The following discussion borrows from \cite{Carroll} and
% \cite{MisnerThorneWheeler}, but mostly from \cite{Wald}. We start by
% putting the words ``homogeneous'' and ``isotropic'' into precise
% language. 
% \begin{itemize}
% \item We define a spacetime as \textit{homogeneous} if we can foliate
%   the spacetime by a one-parameter family of spacelike hypersurfaces
%   $\Sigma_t$ such that, for a given hypersurface $\Sigma_{t_0}$ at
%   time $t_0$ and points $p,q\in\Sigma_{t_0}$, there exists an isometry of
%   the four-metric $g_{\mu\nu}$ that maps $p$ into $q$ \cite{Wald}.
% \item We define a spacetime as \textit{isotropic} if at each point
%   there exits a congruence of timelike curves $\alpha$ with tangent vectors
%   $u^\mu$ such that, for any point $p$ in the spacetime and any two
%   space-like vectors $s_1^\mu$ and $s_2^\mu$ orthogonal to $u^\mu$ at
%   $p$, there exists an isometry of $g_{\mu\nu}$ which leaves $p$ and
%   $u^\mu$ at $p$ fixed but changes $s_1^\mu$ and $s_2^\mu$
%   \cite{Wald}.
% \end{itemize}
% Note that if a universe is \textit{both} homogeneous \textit{and}
% isotropic, the foliating hypersurfaces $\Sigma_t$ must be orthogonal
% to the timelike tangent vector field $u^\mu$ \cite{Wald}. Otherwise,
% it is possible to use proof by contradiction to show that the isometry
% guaranteed by isotropy cannot exist \cite{Wald}. This gives us a piece
% of a canonical coordinate system to use. We define the parameter
% indexing our hypersurfaces as time $t$.
% 
% On each spacelike hypersurface in our foliation, we can restrict the
% full Lorentzian four-metric $g_{\mu\nu}$ to a Riemannian three-metric
% $h_{ij}$. Because of homogeneity, there must exist a sufficient number
% of isometries of $h_{ij}$ to map every point in the hypersurface into
% every other point \cite{Wald}. Because of isotropy, it must be impo

The following discussion borrows from \cite{Carroll},
\cite{MisnerThorneWheeler}, and \cite{Wald}. However, we mostly follow
\cite{Carroll}. It is possible to rigorously and precisely define the
notions of homogeneous and isotropic \cite{Wald}. However, one can
prove that homogenous and isotropic spacetimes are foliated by
maximally symmetric spacelike hypersurfaces
\cite{Carroll}. Intuitively, at least, it's not hard to convince
oneself that homogeneity and isotropy are sufficient to enforce
maximal symmetry. There are three classes of such hypersurfaces all of
constant scalar curvature: spherical spaces, hyperbolic spaces, and
flat spaces. Spherical spaces have constant positive
curvature. Hyperbolic spaces have constant negative curvature, and
flat spaces have zero curvature \cite{Carroll}.

% \begin{figure}[htb]
%   \begin{center}
%     \leavevmode
%     \includegraphics[width=\textwidth]{figures/comparison_of_curvatures.png}
%     \caption{A two-dimensional spacelike hypersurface with positive
%       curvature (left), negative curvature (middle) and no curvature
%       (right).}
%     \label{fig:spacelike:curvatures}
%   \end{center}
% \end{figure}
Let us consider a homogenous, isotropic, four-dimensional Lorentzian
spacetime. We can write the metric as \cite{Carroll}
\begin{equation}
  \label{eq:hom:iso:most:general}
  ds^2 = -dt^2 + a^2(t) d\sigma^2,
\end{equation}
where $t$ is the time coordinate, $a(t)$ is a scalar function known as
the \textit{scale factor}, and $d\sigma^2$ is the metric of the
maximally symmetric spacelike hypersurface. Up to some overall scaling
(which we can hide in $a(t)$, we have \cite{Wald}
\begin{equation}
  \label{eq:3-metric}
  d\sigma^2 = \begin{cases}
    d\chi^2 + \sin^2(\chi)(d\theta^2 + \sin^2(\theta)d\phi^2)&\text{for positive curvature}\\
    d\chi^2 + \chi^2(d\theta^2 + \sin^2d\phi^2)&\text{for no curvature}\\
    d\chi^2 + \sinh^2(\chi)(d\theta^2 + \sin^2(\theta)d\phi^2)&\text{for negative curvature}
    \end{cases},
\end{equation}
where $\chi$, $\theta$, and $\phi$ are the usual spherical
coordinates. However, if we choose $r$ such that
\begin{equation}
  \label{eq:def:rho}
  d\chi = \frac{dr}{\sqrt{1 - k r^2}},
\end{equation}
where
\begin{equation}
  \label{eq:k:range}
  k\in\{-1,0,1\}
\end{equation}
is the normalized scalar curvature of the hypersurface, we can
renormalize our equation to put it in a nicer form
\cite{Carroll,MisnerThorneWheeler},
\begin{equation}
  \label{eq:friedmann:equation}
  ds^2 = -dt^2 + a^2(t)\left[\frac{dr^2}{1 - k r^2} + r^2 d\Omega^2\right],
\end{equation}
where
\begin{equation}
  \label{eq:def:domega}
  d\Omega = d\theta^2 + \sin^2(\theta) d\phi^2
\end{equation}
is the angular piece of the metric. This is known as the
\textit{Friedmann-Lemaitre-Roberston-Walker} (FRLW) metric.

\subsection{Evolution and the Friedmann Equation}
\label{subsec:metric:evolution}

The following discussion draws primarily from the lecture notes
\cite{Kempf}, specifically lecture 16. If we calculate the Einstein
tensor
\begin{equation}
  \label{eq:def:einstein:tensor}
  G_{\mu\nu} = R_{\mu\nu} - \frac{1}{2} R g_{\mu\nu}
\end{equation}
(using, e.g., a tool like Maple's differential geometry package
\cite{Maple}) in the frame induced by our choice of coordinates, we
find that it is diagonal \cite{Kempf}:\footnote{We write the tensor as
  a linear operator with one raised index because this produces the
  nicest form of the equations.}
\begin{eqnarray}
  \label{eq:einstein:tensor:calculated:tt}
  G^t_{\ t} &=& -3 \frac{k + \dot{a}^2}{a^2}\\
  \label{eq:einstein:tensor:calculated:space}
  G^i_{\ i} &=& - \frac{k + 2a\ddot{a} + \dot{a}^2}{a^2} \ \forall\ i\in\{1,2,3\},\\
  \label{eq:einstein:tensor:off:diagonal}
  G^{\mu}_{\ \nu} &=& 0\  \forall\ \mu\neq\nu
\end{eqnarray}
where we have suppressed the $t$-dependence of $a$ and where $\dot{a}$
is a time-derivative of $a$ in the usual way. 

If we feed the Einstein tensor into Einstein's equation in geometric units,
\begin{equation}
  \label{eq:einsteins:equation:1}
  G_{\mu \nu} = 8\pi \tilde{T}_{\mu\nu} - \Lambda g_{\mu\nu},
\end{equation}
since $g_{\mu\nu}$ is diagonal, we see that $T^\mu_{\ \nu}$ must be
diagonal too. Indeed, with one index raised, it looks like the
energy-momentum tensor for a perfect fluid in its rest frame:
\begin{eqnarray}
  T^{\mu}_{\ \nu} &=& \tilde{T}^{\mu}_{\ \nu} - \frac{\Lambda}{8\pi}\mathcal{I}\nonumber\\
  \label{eq:T:mu:nu}
  &=& \left[\begin{array}{cccc}-\rho&0&0&0\\0&p&0&0\\0&0&p&0\\0&0&0&p\end{array}\right] - \frac{\Lambda}{8\pi} \left[\begin{array}{cccc}1&0&0&0\\0&1&0&0\\0&0&1&0\\0&0&0&1\end{array}\right],
\end{eqnarray}
where $\mathcal{I}$ is the identity operator, $\rho$ is the density of
the fluid, and $p$ is the pressure of the fluid. $\tilde{T}^0_{\ 0}$
is negative because we've raised one index. Obviously, this is just a
mathematical coincidence, and we've even manipulated our definition of
pressure to hide our choice of coordinates. However, one can make weak
physical arguments to justify it \cite{Carroll,MisnerThorneWheeler}.

The time-time component of Einstein's equation
\eqref{eq:einsteins:equation:1} then gives the first Friedmann equation \cite{Carroll,MisnerThorneWheeler,Kempf}:
\begin{equation}
  \label{eq:Friedmann:1}
  \frac{\dot{a}^2 + k}{a^2} = \frac{8\pi \rho + \Lambda}{3}.
\end{equation} 
Any space-space diagonal component gives the second Friedmann equation
\cite{Carroll,MisnerThorneWheeler,Kempf}:
\begin{equation}
  \label{eq:Friedmann:2}
  - \frac{k + 2a\ddot{a} + \dot{a}^2}{a^2} = 8\pi p + \Lambda.
\end{equation}
To use these equations to describe an FLRW universe, we need one more
piece of information: the relationship between density and pressure,
called an equation of state
\cite{Carroll,MisnerThorneWheeler,Kempf}. We parametrize our ignorance
of the equation of state as 
\begin{equation}
  \label{eq:def:eos}
  p = \omega(\rho) \rho,
\end{equation}
where $\omega(\rho)$ is some scalar function. 

In the simple case when $\omega$ is a constant, we get several
different regimes. In the case of $\omega \geq 0$, as for a
radiation-dominated or matter-dominated universe
\cite{MisnerThorneWheeler,Carroll,Kempf},\footnote{If $\omega=0$, we
  have a matter-dominated universe where the density scales as one
  over the scale factor cubed, in other words, one over volume. If
  $\omega=1/3$, we have a radiation-dominated universe, where the
  density scales as one over the scale factor to the fourth. The extra
  scaling of the scale factor is due to cosmological redshift
  \cite{Carroll,Kempf}.} the scale factor increases monotonically and
at a decelerating rate \cite{MisnerThorneWheeler,Carroll,Kempf}. In
the case of $\omega = -1$, one gets a dark-energy dominated universe
with accelerating expansion
\cite{MisnerThorneWheeler,Carroll,Kempf}. And if $\omega$ is simply
very close to $-1$, one gets an inflationary universe
\cite{MisnerThorneWheeler,Carroll,Kempf}. As a test, will numerically
explore the cases when $\omega$ is constant and we have a known
regime. We will also numerically explore the cases when $\omega$ is a
more sophisticated object.

\section{Numerical Approach}
\label{sec:numerical:approach}

In any numerical problem, there are two steps. The first is two frame
the problem in a way compatible with numerical methods. The second
step is to actually design and implement a method. To limit our
problem domain, from here on out, we assume that
\begin{equation}
  \label{eq:def:k}
  k = 0.
\end{equation}

\subsection{Framing the Problem}
\label{subsec:framing:the:problem}

We will use a numerical algorithm to solve initial value
problems. These algorithms are usually formulated as first order ODE
solvers. The ode is written in the form
\begin{equation}
  \label{eq:ode:formulation}
  \diff{\myvec{y}}{t} = \myvec{f}(y,t)\text{ with }\myvec{y}(0) = \myvec{y}_0,
\end{equation}
where $\myvec{y}$ is a vector containing all the functions of $t$ one
wishes to solve for. Let's see if we can't write the Friedmann
equations in such a form. 

Our first step is to reduce the second-order ODE system to a
first-order system. We define a new variable, $b(t)$ such that
\begin{equation}
  \label{eq:def:b}
  \dot{a}(t) = b(t).
\end{equation}
Our next step is to remove the variable $\rho$ from our system. To do
so, we assume that the density $\rho(t)$ can be written as a function
of the scale factor $a$ and its derivatives. Fortunately, we can solve
equation \eqref{eq:Friedmann:1} for $\rho$ \cite{Kempf} to find:
\begin{equation}
  \label{eq:rho:of:a}
  \rho(a,b) = \frac{3 b - \Lambda a^2}{8\pi a^2},
\end{equation}
where we have substituted equation \eqref{eq:def:b} and used equation
\eqref{eq:def:k}. Finally, we solve equation \eqref{eq:Friedmann:2}
for $\dot{b}=\ddot{a}$:
\begin{equation}
  \label{eq:derivative:b}
  \dot{b} = - \frac{b^2 + 8\pi \omega \rho a^2 + \Lambda a^2}{2a}.
\end{equation}

So, putting it all together, if we define the two-vector
\begin{equation}
  \label{eq:def:y}
  \myvec{y}(t) = \twovector{a(t)}{b(t)},
\end{equation}
then we can write the equations governing the evolution of the
universe as
\begin{equation}
  \label{eq:y:prime}
  \ddt \myvec{y}(t) = \myvec{f}(y)
  = \twovector{b}{- \dfrac{b^2 + 8\pi \omega(\rho(a,b)) \rho(a,b) a^2 + \Lambda a^2}{2a}},
\end{equation}
where $\rho(a,b)$ is given in equation \eqref{eq:rho:of:a} and
$\omega(\rho(a,b))$ is defined by the $\omega(\rho)$ that we choose.

\subsection{Initial Data}
\label{subsec:initial:data}

Even with evolution equations, we're still missing some critical
information. A first-order ODE system needs one initial value for each
unknown function.

Obviously, equation \eqref{eq:y:prime} breaks down when the scale
factor is zero. Therefore we must be careful to avoid starting a
simulation too close to the big bang singularity. For this reason, we
will always assume that $a(0) = 1$. However, we still need to choose
an initial value for $b$. This depends on the energy density in the
initial universe. We will play around with this value as we explore
our numerical solutions. However, since observations tell us the
universe is expanding, we will usually assume that $b(0) \geq 0$.

\subsection{Numerical Method}
\label{subsec:numerical:method}

For a given $\omega(\rho)$ and set of initial data, we solve the ODE
system using a ``Runge-Kutta'' algorithm. Before we define
Runge-Kutta, let's first describe a simpler, similar, method. The
definition of a derivative is
\begin{equation}
  \label{eq:derivative:definition}
  \ddt\myvec{y}(t) = \lim_{h\to 0}\left[\frac{\myvec{y}(t+h) - \myvec{y}(t)}{h}\right].
\end{equation}
Or, alternatively, if $h$ is sufficiently small,
\begin{equation}
  \label{eq:forward:euler}
  \myvec{y}(t+h) = \myvec{y}(t) + h \dt{\myvec{y}}(t).
\end{equation}
If we know $\myvec{y}(t_0)$ and $\ddt\myvec{y}(t_0)$, then we can use
equation \eqref{eq:forward:euler} to solve for $\myvec{y}(t+h)$. Then,
let $t_1 = t_0 + h$ and use equation \eqref{eq:forward:euler} to solve
for $\myvec{y}(t_1 + h)$. In this way, we can solve for $\myvec{y}(t)$
for all $t > t_0$. This method is called the ``forward Euler'' method
\cite{Heath}.

Runge-Kutta methods are more sophisticated. One can use a Taylor
series expansion to define a more accurate iterative scheme that
relies on higher-order derivatives, not just first
derivatives. However, since we only have first derivative information,
we simulate higher-order derivatives by evaluating the first
derivative at a number of different values of $t$. Then, of course,
the higher-order derivatives are finite differences of these
evaluations \cite{NumericalRecipes,Heath}.

We use a fourth-order Runge-Kutta method---which means the method
effectively incorporates the first four derivatives of a
function---with adaptive step sizes: the 4(5) Runge-Kutta-Fehlberg
method \cite{Fehlberg}. We use our own implementation, written in
C++. The Runge-Kutta solver by itself can be found here:
\cite{RKF45}.\footnote{It is good coding practice to make one's code as
  flexible as possible. By separating the initial-value solver from a
  specific application, we make it possible to reuse our code.} In the
context of FLRW metrics, our implementation can be found here:
\cite{FLRWCode}. Both pieces of code are open-sourced, and the reader
should feel free to download and explore either program.

\section{Results}
\label{sec:results}

\section{Conclusion}
\label{sec:conclusion}

%Bibliography
\bibliography{frlw_project}
\bibliographystyle{hunsrt}

\end{document}