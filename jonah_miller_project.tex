\documentclass[]{article}
% Author: Jonah Miller (jonah.maxwell.miller@gmail.com)
% Time-stamp: <2013-12-14 21:18:33 (jonah)>

% packages
\usepackage{fullpage}
\usepackage{amsmath}
\usepackage{amssymb}
\usepackage{latexsym}
\usepackage{graphicx}
\usepackage{mathrsfs}
\usepackage{verbatim}
\usepackage{braket}
\usepackage{listings}
\usepackage{pdfpages}
\usepackage{listings}
\usepackage{color}
\usepackage{hyperref}

% for the picture environment
\setlength{\unitlength}{1cm}

%preamble
\title{Numerical Explorations of FLRW Cosmologies}

\author{Jonah Miller\\
  \textit{jonah.maxwell.miller@gmail.com}} 

\date{General Relativity for Cosmology\\Achim Kempf\\ Fall 2013}

% Macros
\newcommand{\R}{\mathbb{R}} % Real number line
\newcommand{\N}{\mathbb{N}} % Integers
\newcommand{\eval}{\biggr\rvert} %evaluated at
\newcommand{\myvec}[1]{\vec{#1}} % vectors for me
% total derivatives 
\newcommand{\diff}[2]{\frac{d #1}{d #2}} 
\newcommand{\dd}[1]{\frac{d}{d #1}}
% partial derivatives
\newcommand{\pd}[2]{\frac{\partial #1}{\partial #2}} 
\newcommand{\pdd}[1]{\frac{\partial}{\partial #1}} 
% t derivatives
\newcommand{\ddt}{\dd{t}}
\newcommand{\dt}[1]{\diff{#1}{t}}
% A 2-vector
\newcommand{\twovector}[2]{\left[\begin{matrix}#1\\#2\end{matrix}\right]}
% A 3 vector.
\newcommand{\threevector}[3]{\left[\begin{matrix} #1\\ #2\\ #3 \end{matrix}\right]}
% Error function
\newcommand{\erf}{\text{erf}}

\begin{document}

\maketitle

% \begin{abstract}
%   In this paper, we numerically investigate the past and future of the
%   night sky. We briefly discuss the origin of the so-called
%   Friedmann-Lemaitre-Robertson-Walker metric, the homogenous,
%   isotropic matter-filled universe. We numerically explore the
%   evolution of the metric given various equations of state of the
%   fluid filling spacetime.
% \end{abstract}

\section{Introduction}
\label{sec:intro}

To attain a picture of the universe as a whole, we must look at it on
the largest possible scales. Since the universe is almost certainly
much larger than the observable universe, or indeed the universe we
will ever observe
\cite{MapOfUniverse,Carroll,MisnerThorneWheeler,Wald}, some guesswork
is necessary \cite{Wald}. The most popular---assumption
used to study the universe as a whole is the \textit{Copernican
  principle} \cite{BonesOfCopernicus}, the notion that, on the larges
scales, the universe looks the same everywhere and in every direction
\cite{Carroll,Wald}.

In modern language, we assume that the universe is homogeneous and
isotropic. That is, we assume that the energy-momentum tensor that
feeds Einstein's equations is the same everywhere and in space and has
no preferred direction
\cite{Carroll,MisnerThorneWheeler,Wald}. Obviously, this isn't the
only approach one can take. Roughly speaking one can generate a
spacetime with an arbitrary amount of symmetry between no symmetry and
maximal symmetry. The homogeneous isotropic universe is almost as
symmetric as one can get. One can add time-like symmetry to attain de
Sitter, anti de Sitter, or Minkowski spacetimes, but that's as
symmetric as you can get
\cite{Carroll,MisnerThorneWheeler,Wald}. Relaxing the symmetry
conditions on the energy-momentum tensor results in, for example, the
mixmaster universe \cite{MisnerThorneWheeler}. However, observational
evidence, most strongly the cosmic microwave background, suggests that
the universe is homogeneous and isotropic to a very high degree
\cite{Planck}.

In this work, we derive the equations governing the evolution of a
homogeneous isotropic universe and numerically explore how such a
universe could evolve. In section \ref{sec:the:metric} we briefly
derive the so-called Friedmann equations, the equations governing such
spacetimes. In section \ref{sec:numerical:approach}, we briefly
discuss the numerical tools we use to evolve these equations. In
section \ref{sec:results}, we present our results. Finally, in section
\ref{sec:conclusion}, we offer some concluding remarks.

\section{The Friedmann-Lemaitre-Robertson-Walker Metric}
\label{sec:the:metric}

First, we derive the metric of a homogenous, isotropic spacetime. Then
we derive the analytic formulae for its evolution. Because we derived
the Friedmann equations using tetrads in class \cite{Kempf}, we
decided to work in coordinates for another perspective.

\subsection{Deriving the Metric}
\label{subsec:metric:derivation}

% The following discussion borrows from \cite{Carroll} and
% \cite{MisnerThorneWheeler}, but mostly from \cite{Wald}. We start by
% putting the words ``homogeneous'' and ``isotropic'' into precise
% language. 
% \begin{itemize}
% \item We define a spacetime as \textit{homogeneous} if we can foliate
%   the spacetime by a one-parameter family of spacelike hypersurfaces
%   $\Sigma_t$ such that, for a given hypersurface $\Sigma_{t_0}$ at
%   time $t_0$ and points $p,q\in\Sigma_{t_0}$, there exists an isometry of
%   the four-metric $g_{\mu\nu}$ that maps $p$ into $q$ \cite{Wald}.
% \item We define a spacetime as \textit{isotropic} if at each point
%   there exits a congruence of timelike curves $\alpha$ with tangent vectors
%   $u^\mu$ such that, for any point $p$ in the spacetime and any two
%   space-like vectors $s_1^\mu$ and $s_2^\mu$ orthogonal to $u^\mu$ at
%   $p$, there exists an isometry of $g_{\mu\nu}$ which leaves $p$ and
%   $u^\mu$ at $p$ fixed but changes $s_1^\mu$ and $s_2^\mu$
%   \cite{Wald}.
% \end{itemize}
% Note that if a universe is \textit{both} homogeneous \textit{and}
% isotropic, the foliating hypersurfaces $\Sigma_t$ must be orthogonal
% to the timelike tangent vector field $u^\mu$ \cite{Wald}. Otherwise,
% it is possible to use proof by contradiction to show that the isometry
% guaranteed by isotropy cannot exist \cite{Wald}. This gives us a piece
% of a canonical coordinate system to use. We define the parameter
% indexing our hypersurfaces as time $t$.
% 
% On each spacelike hypersurface in our foliation, we can restrict the
% full Lorentzian four-metric $g_{\mu\nu}$ to a Riemannian three-metric
% $h_{ij}$. Because of homogeneity, there must exist a sufficient number
% of isometries of $h_{ij}$ to map every point in the hypersurface into
% every other point \cite{Wald}. Because of isotropy, it must be impo

The following discussion borrows from \cite{Carroll},
\cite{MisnerThorneWheeler}, and \cite{Wald}. However, we mostly follow
\cite{Carroll}. It is possible to rigorously and precisely define the
notions of homogeneous and isotropic \cite{Wald}. However, one can
prove that homogenous and isotropic spacetimes are foliated by
maximally symmetric spacelike hypersurfaces
\cite{Carroll}. Intuitively, at least, it's not hard to convince
oneself that homogeneity and isotropy are sufficient to enforce
maximal symmetry. There are three classes of such hypersurfaces all of
constant scalar curvature: spherical spaces, hyperbolic spaces, and
flat spaces. Spherical spaces have constant positive
curvature. Hyperbolic spaces have constant negative curvature, and
flat spaces have zero curvature \cite{Carroll}.

% \begin{figure}[htb]
%   \begin{center}
%     \leavevmode
%     \includegraphics[width=\textwidth]{figures/comparison_of_curvatures.png}
%     \caption{A two-dimensional spacelike hypersurface with positive
%       curvature (left), negative curvature (middle) and no curvature
%       (right).}
%     \label{fig:spacelike:curvatures}
%   \end{center}
% \end{figure}
Let us consider a homogenous, isotropic, four-dimensional Lorentzian
spacetime. We can write the metric as \cite{Carroll}
\begin{equation}
  \label{eq:hom:iso:most:general}
  ds^2 = -dt^2 + a^2(t) d\sigma^2,
\end{equation}
where $t$ is the time coordinate, $a(t)$ is a scalar function known as
the \textit{scale factor}, and $d\sigma^2$ is the metric of the
maximally symmetric spacelike hypersurface. Up to some overall scaling
(which we can hide in $a(t)$, we have \cite{Wald}
\begin{equation}
  \label{eq:3-metric}
  d\sigma^2 = \begin{cases}
    d\chi^2 + \sin^2(\chi)(d\theta^2 + \sin^2(\theta)d\phi^2)&\text{for positive curvature}\\
    d\chi^2 + \chi^2(d\theta^2 + \sin^2d\phi^2)&\text{for no curvature}\\
    d\chi^2 + \sinh^2(\chi)(d\theta^2 + \sin^2(\theta)d\phi^2)&\text{for negative curvature}
    \end{cases},
\end{equation}
where $\chi$, $\theta$, and $\phi$ are the usual spherical
coordinates. However, if we choose $r$ such that
\begin{equation}
  \label{eq:def:rho}
  d\chi = \frac{dr}{\sqrt{1 - k r^2}},
\end{equation}
where
\begin{equation}
  \label{eq:k:range}
  k\in\{-1,0,1\}
\end{equation}
is the normalized scalar curvature of the hypersurface, we can
renormalize our equation to put it in a nicer form
\cite{Carroll,MisnerThorneWheeler},
\begin{equation}
  \label{eq:friedmann:equation}
  ds^2 = -dt^2 + a^2(t)\left[\frac{dr^2}{1 - k r^2} + r^2 d\Omega^2\right],
\end{equation}
where
\begin{equation}
  \label{eq:def:domega}
  d\Omega = d\theta^2 + \sin^2(\theta) d\phi^2
\end{equation}
is the angular piece of the metric. This is known as the
\textit{Friedmann-Lemaitre-Roberston-Walker} (FRLW) metric.

\subsection{Evolution and the Friedmann Equation}
\label{subsec:metric:evolution}

The following discussion draws primarily from the lecture notes
\cite{Kempf}, specifically lecture 16. If we calculate the Einstein
tensor
\begin{equation}
  \label{eq:def:einstein:tensor}
  G_{\mu\nu} = R_{\mu\nu} - \frac{1}{2} R g_{\mu\nu}
\end{equation}
(using, e.g., a tool like Maple's differential geometry package
\cite{Maple}) in the frame induced by our choice of coordinates, we
find that it is diagonal \cite{Kempf}:\footnote{We write the tensor as
  a linear operator with one raised index because this produces the
  nicest form of the equations.}
\begin{eqnarray}
  \label{eq:einstein:tensor:calculated:tt}
  G^t_{\ t} &=& -3 \frac{k + \dot{a}^2}{a^2}\\
  \label{eq:einstein:tensor:calculated:space}
  G^i_{\ i} &=& - \frac{k + 2a\ddot{a} + \dot{a}^2}{a^2} \ \forall\ i\in\{1,2,3\},\\
  \label{eq:einstein:tensor:off:diagonal}
  G^{\mu}_{\ \nu} &=& 0\  \forall\ \mu\neq\nu
\end{eqnarray}
where we have suppressed the $t$-dependence of $a$ and where $\dot{a}$
is a time-derivative of $a$ in the usual way. 

If we feed the Einstein tensor into Einstein's equation in geometric units,
\begin{equation}
  \label{eq:einsteins:equation:1}
  G_{\mu \nu} = 8\pi \tilde{T}_{\mu\nu} - \Lambda g_{\mu\nu},
\end{equation}
since $g_{\mu\nu}$ is diagonal, we see that $T^\mu_{\ \nu}$ must be
diagonal too. Indeed, with one index raised, it looks like the
energy-momentum tensor for a perfect fluid in its rest frame:
\begin{eqnarray}
  T^{\mu}_{\ \nu} &=& \tilde{T}^{\mu}_{\ \nu} - \frac{\Lambda}{8\pi}\mathcal{I}\nonumber\\
  \label{eq:T:mu:nu}
  &=& \left[\begin{array}{cccc}-\rho&0&0&0\\0&p&0&0\\0&0&p&0\\0&0&0&p\end{array}\right] - \frac{\Lambda}{8\pi} \left[\begin{array}{cccc}1&0&0&0\\0&1&0&0\\0&0&1&0\\0&0&0&1\end{array}\right],
\end{eqnarray}
where $\mathcal{I}$ is the identity operator, $\rho$ is the density of
the fluid, and $p$ is the pressure of the fluid. $\tilde{T}^0_{\ 0}$
is negative because we've raised one index. Obviously, this is just a
mathematical coincidence, and we've even manipulated our definition of
pressure to hide our choice of coordinates. However, one can make weak
physical arguments to justify it \cite{Carroll,MisnerThorneWheeler}.

The time-time component of Einstein's equation
\eqref{eq:einsteins:equation:1} then gives the first Friedmann equation \cite{Carroll,MisnerThorneWheeler,Kempf}:
\begin{equation}
  \label{eq:Friedmann:1:orig}
  \frac{\dot{a}^2 + k}{a^2} = \frac{8\pi \rho + \Lambda}{3}.
\end{equation} 
Any space-space diagonal component gives a second equation
\cite{Carroll,MisnerThorneWheeler,Kempf}:
\begin{equation}
  \label{eq:einstein:2}
  - \frac{k + 2a\ddot{a} + \dot{a}^2}{a^2} = 8\pi p + \Lambda.
\end{equation}
Now we take
\begin{equation}
  \label{eq:trace:condition}
  -\frac{1}{2} a \left[\left(\text{equation \eqref{eq:Friedmann:1:orig}}\right) + \frac{1}{3}\left(\text{equation \eqref{eq:einstein:2}}\right)\right],
\end{equation}
which yields
\begin{equation}
  \label{eq:Friedmann:2:orig}
  \frac{\ddot{a}}{a} = -\frac{4\pi}{3} \left(\rho + 3 p\right) + \frac{\Lambda}{3},
\end{equation}
which is the second Friedmann equation \cite{Kempf}. To use these
equations to describe an FLRW universe, we need one more piece of
information: the relationship between density and pressure, called an
equation of state \cite{Carroll,MisnerThorneWheeler,Kempf}. We
parametrize our ignorance of the equation of state as
\begin{equation}
  \label{eq:def:eos}
  p = \omega(\rho) \rho,
\end{equation}
where $\omega(\rho)$ is some scalar function. 

Finally, we hide the cosmological constant as a type of matter, which
we call dark matter. This corresponds to resetting
\begin{equation}
  \label{eq:rho:transformation}
  \rho\to\rho - \frac{\Lambda}{8\pi}.
\end{equation}
In this case, the Friedmann equations are
\begin{eqnarray}
  \label{eq:Friedmann:1}
  \frac{\dot{a}^2 + k}{a^2} &=& \frac{8\pi}{3}\rho\\
  \label{eq:Friedmann:2}
  \frac{\ddot{a}}{a} &=& -\frac{4\pi}{3} \left(\rho + 3 p\right).
\end{eqnarray}
Then a cosmological constant corresponds to an energy density that
does not dilute with time... and one with negative pressure. Unless
otherwise stated, we will make this choice.

In the simple case when $\omega$ is a constant, we get several
different regimes. In the case of $\omega \geq 0$, as for a
radiation-dominated or matter-dominated universe
\cite{MisnerThorneWheeler,Carroll,Kempf},\footnote{If $\omega=0$, we
  have a matter-dominated universe where the density scales as one
  over the scale factor cubed, in other words, one over volume. If
  $\omega=1/3$, we have a radiation-dominated universe, where the
  density scales as one over the scale factor to the fourth. The extra
  scaling of the scale factor is due to cosmological redshift
  \cite{Carroll,Kempf}.} the scale factor increases monotonically and
at a decelerating rate \cite{MisnerThorneWheeler,Carroll,Kempf}. In
the case of $\omega = -1$, one gets a dark-energy dominated universe
with accelerating expansion
\cite{MisnerThorneWheeler,Carroll,Kempf}. And if $\omega$ is simply
very close to $-1$, one gets an inflationary universe
\cite{MisnerThorneWheeler,Carroll,Kempf}. As a test, will numerically
explore the cases when $\omega$ is constant and we have a known
regime. We will also numerically explore the cases when $\omega$ is a
more sophisticated object.

\section{Numerical Approach}
\label{sec:numerical:approach}

In any numerical problem, there are two steps. The first is two frame
the problem in a way compatible with numerical methods. The second
step is to actually design and implement a method. To limit our
problem domain, from here on out, we assume that
\begin{equation}
  \label{eq:def:k}
  k = 0.
\end{equation}

\subsection{Framing the Problem}
\label{subsec:framing:the:problem}

We will use a numerical algorithm to solve initial value
problems. These algorithms are usually formulated as first order ODE
solvers. The ode is written in the form
\begin{equation}
  \label{eq:ode:formulation}
  \diff{\myvec{y}}{t} = \myvec{f}(y,t)\text{ with }\myvec{y}(0) = \myvec{y}_0,
\end{equation}
where $\myvec{y}$ is a vector containing all the functions of $t$ one
wishes to solve for. Let's see if we can't write the Friedmann
equations in such a form. 

Our first step is to reduce the second-order ODE system to a
first-order system. We will do this by eliminating $\ddot{a}$ in favor
of some other variable. If we differentiate equation
\eqref{eq:Friedmann:1}, we get
\begin{equation}
  \label{eq:dF1}
  -2 \frac{\dot{a}^2}{a^3} + \frac{\ddot{a}}{a^2} = \frac{8\pi}{3}\dot{\rho}.
\end{equation}
We can substitute $\ddot{a}$ in from \eqref{eq:Friedmann:2} to obtain
an equation for the time evolution of $\rho$:
\begin{equation}
  \label{eq:Friedmann:rho}
  \dot{\rho} = - 3 \left(\frac{\dot{a}}{a}\right)\left(\rho + p\right),
\end{equation}
which expresses conservation of mass energy. As a last step, we use
equation \eqref{eq:Friedmann:1} and equation \eqref{eq:def:eos} to
eliminate $\dot{a}$ and $p$ from equation \eqref{eq:Friedmann:rho} to
attain an evolution equation for $\rho$:
\begin{equation}
  \label{eq:rho:evolution}
  \dot{\rho} = \mp 3\left[\rho + \omega(\rho)\rho\right]\sqrt{\frac{8\pi}{3}\rho-k}.
\end{equation}

So, putting it all together, if we define the two-vector
\begin{equation}
  \label{eq:def:y}
  \myvec{y}(t) = \twovector{a(t)}{\rho(t)},
\end{equation}
then we can write the equations governing the evolution of the
universe as
\begin{equation}
  \label{eq:y:prime:orig}
  \ddt \myvec{y}(t) = \myvec{f}(y)
  = \pm\twovector{\sqrt{8\pi a^2\rho/3}}{-3\left(\rho + \omega(\rho)\rho\right)\sqrt{\frac{8\pi}{3}\rho-k}}.
\end{equation}
Since our universe is expanding and we're interested in an expanding
universe, we choose $\dot{a}>0$. This gives us
\begin{equation}
  \label{eq:y:prime}
  \myvec{f}(y) = \twovector{\sqrt{8\pi a^2\rho/3}}{-3\left(\rho + \omega(\rho)\rho\right)\sqrt{\frac{8\pi}{3}\rho-k}}.
\end{equation}

\subsection{Initial Data}
\label{subsec:initial:data}

Even with evolution equations, we're still missing some critical
information. A first-order ODE system needs one initial value for each
unknown function.

Obviously, equation \eqref{eq:y:prime} breaks down when the scale
factor is zero. Therefore we must be careful to avoid starting a
simulation too close to the big bang singularity. For this reason, we
will never assume that $a(0) = 0$. If convenient, we will choose
$a(0)$ to be close to zero. There is one exception, which is the case
of a dark-energy dominated universe, where we will choose $a(0)=1$,
since such a universe is exponentially growing.

We still need to choose a value for $\rho(0)$, however, so long as
$\rho(0) > 0$, we should expect it to not matter much. General
relativity is invariant under an overall re-scaling of energy
\cite{MisnerThorneWheeler,Wald}. However, to get a better idea of
qualitative behavior, we will chose a number of values of $\rho(0)$ in
each case.

\subsection{Numerical Method}
\label{subsec:numerical:method}

For a given $\omega(\rho)$ and set of initial data, we solve the ODE
system using a ``Runge-Kutta'' algorithm. Before we define
Runge-Kutta, let's first describe a simpler, similar, method. The
definition of a derivative is
\begin{equation}
  \label{eq:derivative:definition}
  \ddt\myvec{y}(t) = \lim_{h\to 0}\left[\frac{\myvec{y}(t+h) - \myvec{y}(t)}{h}\right].
\end{equation}
Or, alternatively, if $h$ is sufficiently small,
\begin{equation}
  \label{eq:forward:euler}
  \myvec{y}(t+h) = \myvec{y}(t) + h \dt{\myvec{y}}(t).
\end{equation}
If we know $\myvec{y}(t_0)$ and $\ddt\myvec{y}(t_0)$, then we can use
equation \eqref{eq:forward:euler} to solve for $\myvec{y}(t+h)$. Then,
let $t_1 = t_0 + h$ and use equation \eqref{eq:forward:euler} to solve
for $\myvec{y}(t_1 + h)$. In this way, we can solve for $\myvec{y}(t)$
for all $t > t_0$. This method is called the ``forward Euler'' method
\cite{Heath}.

Runge-Kutta methods are more sophisticated. One can use a Taylor
series expansion to define a more accurate iterative scheme that
relies on higher-order derivatives, not just first
derivatives. However, since we only have first derivative information,
we simulate higher-order derivatives by evaluating the first
derivative at a number of different values of $t$. Then, of course,
the higher-order derivatives are finite differences of these
evaluations \cite{Heath,NumericalRecipes}.

We use a fourth-order Runge-Kutta method---which means the method
effectively incorporates the first four derivatives of a
function---with adaptive step sizes: the 4(5) Runge-Kutta-Fehlberg
method \cite{Fehlberg}. We use our own implementation, written in
C++. The Runge-Kutta solver by itself can be found here:
\cite{RKF45}.\footnote{It is good coding practice to make one's code as
  flexible as possible. By separating the initial-value solver from a
  specific application, we make it possible to reuse our code.} In the
context of FLRW metrics, our implementation can be found here:
\cite{FLRWCode}. Both pieces of code are open-sourced, and the reader
should feel free to download and explore either program.

\section{Results}
\label{sec:results}

Before we delve into unknown territory, it's a good idea to explore
known physically interesting FLRW solutions. This is what we do
now. 

\subsection{Known Solutions}
\label{subsec:known:solutions}

If the universe is dominated by radiation, as in the case in the early
universe, just after reheating \cite{Kempf}, we should expect the
scale factor to grow as $a(t) = \left(t/t_0\right)^{1/2}$
\cite{Kempf}. To account for not quite starting at $a=0$, we fit our
numerical solutions to
\begin{equation}
  \label{eq:radiation:dominated:fit}
  a(t) = a_0 t^{1/2} + b,
\end{equation}
where $a_0$ and be are constants. Figure \ref{fig:radiation:dominated}
shows scale factors for several different choices of $\rho(0)$ and a
fit to an analytic scale factor solution using equation
\eqref{eq:radiation:dominated:fit}, which matches the data extremely
well.\footnote{In all plots showing a fit, we've removed most of the
  data points. There are a minimum of 100 data points generated per
  simulation. However these are hidden to demonstrate the accuracy of
  the fit.}

\begin{figure}[htb]
  \begin{center}
    \leavevmode
    \hspace{-.2cm}
     \includegraphics[width=8.2cm]{figures/radiation_dominated_all_scale_factors.png}
     \hspace{-0.3cm}
     \includegraphics[width=8.3cm]{figures/radiation_dominated_fit_rho0256.png}
     \caption[Radiation Dominated Universe]{The Radiation-Dominated
       Universe. Left: Scale factor as a function of time for a number
       of different values of $\rho(0)$. Right: A fit to equation
       \eqref{eq:radiation:dominated:fit} for $\rho(0)=256$.}
    \label{fig:radiation:dominated}
  \end{center}
\end{figure}

If the universe is dominated by matter, as was the case until
extremely recently \cite{Kempf}, then we expect the scale factor to
scale as $a(t) = \left(t/t_0\right)^{2/3}$ \cite{Kempf}. To account
for not quite starting at $a=0$, we fit our numerical solutions to
\begin{equation}
  \label{eq:matter:dominated:fit}
  a(t) = a_0 t^{2/3} + b.
\end{equation}
Figure \ref{fig:matter:dominated} shows the scale factors and fit.
\begin{figure}[htb]
  \begin{center}
    \leavevmode
    \hspace{-.2cm}
     \includegraphics[width=8.2cm]{figures/matter_dominated_all_scale_factors.png}
     \hspace{-0.3cm}
     \includegraphics[width=8.3cm]{figures/matter_dominated_fit_rho032.png}
     \caption[Matter Dominated Universe]{The Matter-Dominated
       Universe. Left: Scale factor as a function of time for a number
       of different values of $\rho(0)$. Right: A fit to equation
       \eqref{eq:matter:dominated:fit} for $\rho(0)=32$.}
    \label{fig:matter:dominated}
  \end{center}
\end{figure}

If the universe is dominated by dark energy, as will be the case very
soon \cite{Kempf}, then we expect the scale factor to scale as
$a(t)=a_0 e^{t_0t}$. For convenience, we fit to
\begin{equation}
  \label{eq:dark:energy:dominated:fit}
  a(t) = e^{t_0t + b},
\end{equation}
where $t_0$ and $a_0$ are constants. Figure
\ref{fig:dark:energy:dominated} shows the scale factors and fit.
\begin{figure}[htb]
  \begin{center}
    \leavevmode
    \hspace{-.2cm}
     \includegraphics[width=8.2cm]{figures/dark_energy_dominated_universe_scale_factors.png}
     \hspace{-0.3cm}
     \includegraphics[width=8.3cm]{figures/dark_energy_dominated_fit_rho012.png}
     \caption[Matter Dominated Universe]{The Dark Energy-Dominated
       Universe. Left: Scale factor as a function of time for a number
       of different values of $\rho(0)$. Right: A fit to equation
       \eqref{eq:matter:dominated:fit} for $\rho(0)=1.2$.}
    \label{fig:dark:energy:dominated}
  \end{center}
\end{figure}

\subsection{A Multi-Regime Universe}
\label{subsec:multi-regime}

Now that we've confirmed appropriate behavior in each of the
analytically explored regimes, it's time to move into unknown
territory! We will attempt to construct a history of the universe by
splicing together the known solutions to the Friedmann equations in a
smooth way. 

To this end, we let $\omega$ vary smoothly with $\rho$. We want the
radiation, matter, and dark energy dominated regimes of the universe
to emerge naturally, separated by transition regions. Furthermore, we
want $\omega(\rho)$ to be monotonically increasing, since we know that
we first had a radiation dominated universe---radiation scales as
$a^{1/4}$---then a matter dominated universe---matter scales as
$a^{1/3}$---and finally a dark energy dominated universe---dark energy
does not scale with $a$ at all
\cite{Carroll,MisnerThorneWheeler,Kempf}. The dark energy regime
should take over when $\rho$ is very small and the normal matter and
radiation has diluted away \cite{Carroll,MisnerThorneWheeler,Kempf}.

One function that smoothly, asymptotically, and monotonically
interpolates between two constant values is the error function,
\begin{equation}
  \label{eq:def:error:function}
  \erf(x) = \frac{2}{\sqrt{\pi}}\int_0^x e^{-t^2}dt.
\end{equation}
$\erf(-\infty)=-1$ and $\erf(+\infty)=+1$. However, the transition
region is very sharp and the error function is effectively $-1$ for
$x<-2$ and effectively $+1$ for $x>2$. We can therefore create a
smoothly varying $\omega(\rho)$ with the desired properties by taking
a linear combination of error functions:
\begin{equation}
  \label{eq:interpolating:function}
  \omega(\rho) = -1 + \left(\frac{1}{2}\right) \left(1+\frac{1}{3}\right) + \frac{1}{2}\erf\left(\frac{\rho - \rho_{DE}}{\Delta_{DE}}\right) + \left(\frac{1}{2}\right)\left(\frac{1}{3}\right)\erf\left(\frac{\rho - \rho_{RM}}{\Delta_{RM}}\right),
\end{equation}
where $\rho_{DE}$ is the value of $\rho$ marking the change of regimes
from a matter-dominated to a dark energy dominated universe and
$\Delta_{DE}$ is the width of the transition region. $\rho_{RM}$ and
$\Delta_{RM}$ fill analogous roles for the transition from a radiation
to a matter dominated universe.

By varying $\rho_{DE}$, $\rho_{RM}$, $\Delta_{DE}$, and $\Delta_{RM}$,
we can select a function where the transition between regimes is very
abrupt or where the transition is very gradual. We chose four such
functions, as shown in figure \ref{fig:omega:of:rho}: an equation of
state with abrupt transitions
($\rho_{DE}=10,\Delta_{DE}=\Delta_{RM}=1,\rho_{RM}=20$), one with
moderate transitions
($\rho_{DE}=20,\Delta_{DE}=\Delta_{RM}=5,\rho_{RM}=40$), one with
gradual transitions
($\rho_{DE}=25,\Delta_{DE}=\Delta_{RM}=12,\rho_{RM}=80$). And,
finally, as a check, we chose a function with no well defined epochs
at all ($\rho_{DE}=37,\Delta_{DE}=\Delta_{RM}=20,\rho_{RM}=92$).

\begin{figure}[tbh]
  \begin{center}
    \leavevmode
    \includegraphics[width=12cm]{figures/multi_regime_eos_1.png}
    \caption[EOS]{The equation of state variable $\omega$ (see
      equation \eqref{eq:def:eos}) as a function of energy density
      $\rho$, as defined in equation
      \eqref{eq:interpolating:function}. We choose four possible
      scenarios: an equation of state with abrupt transitions
      ($\rho_{DE}=10,\Delta_{DE}=\Delta_{RM}=1,\rho_{RM}=20$), one
      with moderate transitions
      ($\rho_{DE}=20,\Delta_{DE}=\Delta_{RM}=5,\rho_{RM}=40$), one
      with gradual transitions
      ($\rho_{DE}=25,\Delta_{DE}=\Delta_{RM}=12,\rho_{RM}=80$), and
      one with almost no transitions at all
      ($\rho_{DE}=37,\Delta_{DE}=\Delta_{RM}=20,\rho_{RM}=92$)}
    \label{fig:omega:of:rho}
  \end{center}
\end{figure}

The evolution of $a(t)$ for these four choices of $\omega(\rho)$ is
shown in figure \ref{fig:a:varying:omega}, where we also show the
evolution of $\rho$ and $p$ on the same plots. (We have rescaled
$a(t)$ so that it fits nicely with the other variables.) In these
plots, we can see hints of each of the three analytically predicted
epochs of growth. Initially $\ddot{a} < 0$, but it transitions to
$\ddot{a}>0$.

Surprisingly, so long as we allow $\omega(\rho)$ to
vary from $+1/3$ to $0$ to $-1$, $a(t)$ seems remarkably resilient to
the details such as the abruptness of the transition or the precise
location of the transition region. Indeed, when we plotted all four
scale factor functions on the same scale, they overlapped completely.

We postulate that his unexpected resilience is due to the $a^{-2}$
term in the formula for $\dot{\rho}$. Although different fixed values
of $\omega$ drive $a(t)$ at different rates, $a(t)$ in turn drives
$\rho(t)$ into a regime that balances out this change. This stability
is a boon to cosmologists, since it means that, for the purpose of
predictions a piecewise function composed of the analytic solutions is
sufficient---we don't need to look for a more detailed or accurate
equation of state.

\begin{figure}[htb]
  \begin{center}
    \leavevmode
    \hspace{-.2cm}
     \includegraphics[width=8.35cm]{figures/multi_regime_abrupt_transition.png}
     \hspace{-0.3cm}
     \includegraphics[width=8.35cm]{figures/multi_regime_moderate_transition.png}\\
     \vspace{0.2cm}
     \hspace{-.2cm}
     \includegraphics[width=8.35cm]{figures/multi_regime_mild_transition.png}
     \hspace{-0.3cm}
     \includegraphics[width=8.35cm]{figures/multi_regime_no_transition.png}
     \caption[$a(t)$ with a variable $\omega(\rho)$]{$a(t)$ with a
       variable $\omega(\rho)$. We plot the scale factor for the four
       equations of state show in figure \ref{fig:omega:of:rho}.}
     \label{fig:a:varying:omega}
  \end{center}
\end{figure}

\subsection{$\omega \approx -1.2$}
\label{subsec:label:omega:1.2}

The Pan-STARRS1 survey \cite{Pan-STARRS1} predicts that $\omega
\approx -1.2$, which is a very strange type of matter indeed. When
$\omega=-1$, it emerges from the cosmological constant term and the
energy density $\rho$ stays constant as the scale factor changes. If
$\omega<-1$, then the energy density must \textit{increase} with the
scale factor! Since we expect energy density to \textit{dilute} with
increased scale factor \cite{Carroll,Kempf}, this is decidedly odd!

We can test this hypothesis with our FLRW simulation code. And,
indeed, the energy density slowly grows with scale factor, as shown in
figure \ref{fig:omega:1.2}. The result is that the scale factor grows
with time more quickly than an exponential. We fit $\ln(a(t))$ to a
polynomial and found a cubic to fit well. Figure \ref{fig:omega:1.2}
shows the results of one such fit.
\begin{figure}[htb]
  \begin{center}
    \leavevmode
    \hspace{-.2cm}
     \includegraphics[width=8.2cm]{figures/our_future_universe_all_variables_rho035.png}
     \hspace{-0.3cm}
     \includegraphics[width=8.3cm]{figures/our_future_universe_fit_rho035.png}
     \caption[Our Future Universe?]{Our future universe? We solve for
       the universe when $\omega \approx -1.2$. Left: Scale factor,
       density, and pressure as a function of time. Right: A fit to
       equation $a(t) = \exp(a_0 t^3 + b t^2 + c t + d)$. In both
       cases, $\rho(0) = 0.35$.}
    \label{fig:omega:1.2}
  \end{center}
\end{figure}

\section{Conclusion}
\label{sec:conclusion}

In summary, given that the universe is foliated by homogeneous and
isotropic hypersurfaces, we have re-derived how to solve for the
evolution of space as a function of time
\cite{Carroll,MisnerThorneWheeler,Wald,Kempf}. We have also used a
fourth-order Runge-Kutta-Fehlberg method \cite{Fehlberg,RKF45} to
numerically solve for the universe given these evolution equations. We
have reproduced the analytically known epochs of evolution and solved
for the universe as it transitions between analytically known
radiation-dominated, matter-dominated, and dark energy-dominated
regimes. Finally, we have solved for a universe dominated by an
equation of state where $\omega<-1$ and the energy density increases
slowly with scale factor. In this case, we discovered faster than
exponential growth, which agrees qualitatively with expectations.

It is a pleasant surprise to find that the evolution of the universe
extremely resilient to a choice of equation of state,
$p=\omega(\rho)\rho$, so long as $\omega(\rho)$ varies smoothly and
monotonically from $\omega=-1$ to $\omega=+1/3$. This means that it is
sufficient for cosmologists to study the evolution of the universe
epoch by epoch, and not worry too much about the transitional periods.

We would have liked to study the big bounce scenario wherein the scale
factor shrinks to some minimum value and then grows
monotonically. This is certainly analytically feasible. In the case of
a positively curved spacelike foliation dominated by dark energy, for
example, we analytically attain de Sitter space, where $a(t)\approx
\cosh(t)$. Unfortunately, $\dot{a}$ must equal zero in this scenario,
and this condition makes the first-order version of the Friedmann
equations singular. A numerical treatment of the big bounce seems
difficult if we continue to treat it as an initial value problem. A
better approach might be to treat the big bounce scenario as a
boundary value problem with initial and final values of $a(t)$ and a
negative value for $\dot{a}(t)$. In general, the numerical methods to
treat initial value problems are very different from those used to
treat boundary value problems \cite{Heath}. Thus it would be necessary
to implement another approach (e.g., finite elements) anew. This would
be an interesting future project.

%Bibliography
\bibliography{frlw_project}
\bibliographystyle{hunsrt}

\end{document}